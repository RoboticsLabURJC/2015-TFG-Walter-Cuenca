% Appendix Template

\chapter{Appendix Title Here} % Main appendix title

\label{AppendixX} % Change X to a consecutive letter; for referencing this appendix elsewhere, use \ref{AppendixX}

\section{Models}
Se muestra las clases que se han creado al igual que los campos que lo forman, como mencionamos anteriormente esto corresponde con las tablas de nuestra base de datos.
\subsection{Canciones}
\begin{lstlisting}[
frame=single,
commentstyle=\color{CadetBlue},
captionpos=b,
caption=Incluir elementos multimedia remotos l.]
class cancion(models.Model):
 name=models.CharField(max_length=100,null=True)
 song_file=models.FileField(upload_to = 'musica/', max_length=200,null=True)
 def __unicode__(self):
  return self.name
\end{lstlisting}

\subsection{Discos}
\begin{lstlisting}[
frame=single,
commentstyle=\color{CadetBlue},
captionpos=b,
caption=Incluir elementos multimedia remotos l.]
class disco(models.Model):
	name = models.CharField(max_length=100,null=True)
	genero = models.CharField(max_length=50,null=True)
	linkImg = models.ImageField(upload_to="img_disco/",null=True)
	name_cancion = models.ManyToManyField(cancion)
	def __unicode__(self):
		return self.name
\end{lstlisting}

\subsection{Videos}
\begin{lstlisting}[
frame=single,
commentstyle=\color{CadetBlue},
captionpos=b,
caption=Incluir elementos multimedia remotos l.]
class Videos(models.Model):
 TYPE_FILE = (
        ('iframe', 'iframe'),
        ('video', 'video'),
    )
 name = models.CharField(max_length=100,null=True)
 tipo=models.CharField(max_length=6, choices=TYPE_FILE,null=True)
 linkVideo = models.FileField(upload_to = 'video/', max_length=200,null=True,blank=True)
 linkweb = models.URLField(max_length=100,null=True,blank=True)
 def __unicode__(self):
		return self.name
\end{lstlisting}

\subsection{Imagenes}
\begin{lstlisting}[
frame=single,
commentstyle=\color{CadetBlue},
captionpos=b,
caption=Incluir elementos multimedia remotos l.]
class Imagenes(models.Model):
 name = models.CharField(max_length=100,null=True)
 linkImg = models.ImageField(upload_to="images/",null=True)
 def __unicode__(self):
	return self.name
\end{lstlisting}

\subsection{Artista}
\begin{lstlisting}[
frame=single,
commentstyle=\color{CadetBlue},
captionpos=b,
caption=Incluir elementos multimedia remotos l.]
class Artista(models.Model):
 name = models.CharField(max_length=100,null=True)
 linkImg = models.ImageField(upload_to="img_artist/",null=True)
 genero = models.CharField(max_length=100,null=True)
 web = models.URLField(max_length=100,null=True)
 disco = models.ManyToManyField(disco)
 videos = models.ManyToManyField(Videos) 
 galeria = models.ManyToManyField(Imagenes)
 def __unicode__(self):
	return self.name
\end{lstlisting}


\subsection{Tickets}
\begin{lstlisting}[
frame=single,
commentstyle=\color{CadetBlue},
captionpos=b,
caption=Incluir elementos multimedia remotos l.]
class Tickets(models.Model):
	TYPE_TICKET = (
        ('Vip', 'VIP'),
        ('General', 'Estandar'),
    )
	name=models.CharField(null=True,max_length=120)
	precio=models.FloatField(null=True)
	cantidad=models.IntegerField(null=True)
	tipo=models.CharField(max_length=7, choices=TYPE_TICKET ,null=True)
	info=models.CharField(null=True,max_length=120)
	stock=models.BooleanField(default=True)
\end{lstlisting}

\subsection{Contacta}
\begin{lstlisting}[
frame=single,
commentstyle=\color{CadetBlue},
captionpos=b,
caption=Incluir elementos multimedia remotos l.]
class Contacta(models.Model):
	TYPE_DEPARTAMENT = (
        ('E', 'Evento'),
        ('T', 'Ticket'),
				('S', 'Tecnico'),
    )
	nombre=models.CharField(null=True,max_length=120)
	email=models.EmailField(null=True)
	area=models.CharField(max_length=1, choices=TYPE_DEPARTAMENT ,null=True)
	telefono=models.CharField(null=True,max_length=120)
	motivo=models.CharField(null=True,max_length=120)
 	time = models.DateTimeField(default=datetime.datetime.now)
	texto=models.CharField(null=True,max_length=300)
\end{lstlisting}

\subsection{Evento}
\begin{lstlisting}[
frame=single,
commentstyle=\color{CadetBlue},
captionpos=b,
caption=Incluir elementos multimedia remotos l.]
class Evento(models.Model):
	name = models.CharField(max_length=100,null=True)
	cartel = models.ImageField(upload_to="images/",null=True)
	artistas = models.ManyToManyField(Artista)
	imagenes = models.ManyToManyField(Imagenes)
	videos = models.ManyToManyField(Videos)
	entradas =  models.ManyToManyField(Tickets)
	longitud = models.FloatField(null=True)
	latitud = models.FloatField(null=True)
	direcccion = models.CharField(max_length=100,null=True)
	fecha = models.DateTimeField(null=True)
	def __unicode__(self):
		return self.name
\end{lstlisting}

\subsection{Buy}
\begin{lstlisting}[
frame=single,
commentstyle=\color{CadetBlue},
captionpos=b,
caption=Incluir elementos multimedia remotos l.]
class Buy(models.Model):
	typeProducto=models.CharField(max_length=100,null=True) 
	nameProduct=models.ManyToManyField(Evento) 
	typeTicket=models.CharField(max_length=100,null=True) 
	cantidad=models.IntegerField(null=True)
	total=models.IntegerField(null=True)
	def __unicode__(self):
		return self.typeProducto
\end{lstlisting}

\subsection{UserProfile}
\begin{lstlisting}[
frame=single,
commentstyle=\color{CadetBlue},
captionpos=b,
caption=Incluir elementos multimedia remotos l.]
class UserProfile(models.Model):
	user = models.ForeignKey(User, unique=True)
	imgPerfil = models.ImageField(upload_to="img_perfil/",null=True,
    					default='images/Img_Default.jpg')
	sexo = models.CharField(max_length=10,blank=True)
	compra = models.ManyToManyField(Buy)
\end{lstlisting}

