\chapter{Conclusiones}
Este capitulo recoge las conclusiones y futuros trabajos una vez los desarrollos del TFG han concluido y poniendo en balance si se los requisitos iniciales han sido cubiertos.
\section{Conclusiones}
Si echamos una mirada atrás se puede ver que se ha creado satisfactoriamente cada uno de los enunciados y el desarrollo de las practicas destinadas al entorno docente en las que utilizamos múltiples tecnologías web que se marco como objetivo principal. Dentro de este objetivo nos marcamos cuatro subjetivos, los cuáles también hemos cumplido:
\begin{enumerate}
    \item La aplicación del juego del ComeCocos se ha conseguido crear el juego a través de tecnologías Web del clientes sin necesidad de plugins externos.
    \item La aplicación multijugador del ComeCocos se ha conseguido implementar la comunicación entre el servidor del juego y los clientes permitiendo establecer una partida entre jugadores satisfactoriamente.
    \item La aplicación de un sitio Web de una tienda se ha desarrollado con Django y utilizando como BBDD MySQL.
    \item La aplicación de Videoconferencia con WebRTC se ha desarrollado correctamente permitiendo a los usuarios conectarse entre si a través de un navegador permitiendo intercambiar flujo de vídeo, audio o ficheros por medio de tecnologías Web. 
\end{enumerate}

Se puede encontrar tanto esta memoria, como el repositorio del código, vídeos, explicaciones, ejemplos y resultados obtenidos en la mediawiki oficial del proyecto\footnote{\url{http://jderobot.org/Walter-tfg}}
\section{Trabajos futuros}
Este TFG habré las puertas al uso de tecnologías Web de ultima generación. A medida que los desarrollos fueron avanzando se veían puntos de mejora en cada una de las practicas ya sea de forma visual o aportando mayor funcionalidad aunque no influida en llegar a los objetivos marcados por lo que la mejora de cada una de las practicas puede ser una linea de futuros trabajos.

Aunque una linea de trabajo atractivo para el futuro seria fusionar lo aprendido con estas practicas para crear un proyecto que abarque estas tecnologías en un solo desarrollo. Esto podría ser un juego multijugador como el ComeCocos en el que se utiliza en el lado del cliente JS y HTML5 y como mecanismos de comunicación multimedia WebRTC apoyándonose en su canal de datos para información del juego entre los usuarios además que de esta forma permitiría a los jugadores verse mientras hacen uso de la aplicación. A esto habría que sumarle la utilización de WebSockets en la fase inicial de WebRTC y enriquecer la apariencia del juego con WebGL.
