\chapter{VideoConferencia usando WebRTC}
\section{Enunciado}
Las aplicaciones multimedia en tiempo real permiten conectar a distintos usuarios a través de la red e intercambiando información de audio,vídeo u otro tipo. Un ejemplo de este tipo de aplicación es Skype que ha tenido mucho éxito ya que permite a sus usuarios establecer videollamadas e intercambiar información una vez se haya instalado el software.
\begin{figure}[!h]
\begin{center}
   \includegraphics[width=0.9\linewidth]{Figures/skype}
	\decoRule
	\caption[Ejemplo sitio Web]{Skype interfaz videollamada.}
\label{fig:canvasPrimitivas}
\end{center}
\end{figure}

Por lo tanto en esta ultima practica se pide desarrollar una aplicación web similar a Skype que permita a los usuarios conectarse entre si a través de la url de la aplicación.
\paragraph{Requisitos}
Los usuario dentro de la aplicación podrán seleccionar los elementos multimedia(audio y vídeo) que desean compartir,   ademas de seleccionar o crear una sala donde se establecerá la videoconferencia. 

los participantes  de la videoconferencia en la sala  dispondrán de un chat y de la posibilidad de intercambiar ficheros por medio de WebRTC ya que la comunicación sera Peer-to-Peer en este punto.

Para llevar acabo lo descrito anteriormente sera necesario crear un servidor de señalizacion para establecer la fase  de comunicación inicial de WebRTC.
\paragraph{Tecnologías}
Sera necesario emplear WebRTC, API File, JavaScript para el lado cliente mientras que para el servidor de señalizacion utilizamos NodeJS. Ademas se utilizara WebSockets como mecanismo de comunicación con el servidor y Boostratps para trabajar la apariencia de la aplicación.
\section{Diseño}
%%%%%%%% Servidor Desarrollo %%%%%%%%
\section{Desarrollo Servidor Señalizaion}
El primer paso para crear el servidor es importar la librería \textbf{node-static},\textbf{http} y \textbf{socket.io}. La librería \textbf{http} permite crear un servidor a través del método \textbf{createServer} en el puerto 8181 en nuestro caso mientras que la librería \textbf{node-static} permite enviar el documento a los usuarios que se conecten.
\\Finalmente, la librería \textbf{socket.io} se utiliza para establecer comunicación entre el cliente y servidor, figura \ref{fig:EjecucionServer}.
\begin{lstlisting}[
caption=Creación Servidor de Señalizacion.]
 var static = require('node-static');
 var http = require('http');
 var file = new(static.Server)();
 var app = http.createServer(function (req, res) {
  file.serve(req, res);
 }).listen(8181);
 var io = require('socket.io').listen(app);
\end{lstlisting}
\begin{figure}[!h]
\begin{center}
   \includegraphics[width=0.8\linewidth]{Figures/InicioServidor}
	\decoRule
	\caption[ServidorSeñalizacion]{Ejecución Servidor de Señalizacion APP.}
\label{fig:EjecucionServer}
\end{center}
\end{figure}
El siguiente paso es permitir al servidor gestionar adecuadamente cada conexión que se establece a través de socket.io.Por ello, a través del método 'conected' mantendremos conectado.
\\Primero se define el evento \textbf{InfoRoom} que gestiona las peticiones sobre las salas disponibles.A esta petición el servidor contesta con un mensaje \textbf{ReplayInfoRoom} con la lista de las salas en la variable \textbf{listRoom}, figura \ref{fig:EjecucionInfoRoom}.
\begin{lstlisting}[,
caption=Request/Replay lista de salas existentes.]
 socket.on('infoRoom',function(name) {
  socket.emit('ReplayInfoRoom',listRooom);
 });
\end{lstlisting}
\begin{figure}[!h]
\begin{center}
   \includegraphics[width=0.8\linewidth]{Figures/InfoRoomServer}
	\decoRule
	\caption[Request/Replay Salas Servidor]{Request/Replay lista de salas del Servidor.}
\label{fig:EjecucionInfoRoom}
\end{center}
\end{figure}
Tras el envió de la lista de salas disponibles el servidor recibirá un mensaje ..Definimos el evento \textbf{Stablish\_connect} que recibe el nombre y sala a la que el usuario desea conectarse. Con esta información comprueba si el nombre de sala existe a través de la función \textbf{getRoom(nameRoom)} en caso de no existir se guardara en la lista con la función \textbf{setRoom()}.
\\Ademas, comprueba que la sala no este completa para enviar un mensaje \textbf{CreateStream} con el identificador de conexión y un mensaje  \textbf{NewJoined} al resto de usuario de la sala para que conozcan la existencia del nuevo miembro, figura \ref{fig:EjecucionStablishConnection}.
\begin{lstlisting}[
caption=Request/Replay del establecimiento de conexion.]
 socket.on('stablish_connection',function(name,room){
  if(!getRoom(room)){
    setRoom(room,'');
  };
  var numClients = io.sockets.clients(room).length;
  if(numClients < 3){
   socket.username = name;
   socket.room =room;
   socket.join(room);
   socket.emit('CreateStream',socket.id);
   socket.broadcast.to(room).emit('New_Joined',socket.id);
  }else{
   socket.emit('RejectStream',socket.id);
  }
 });
\end{lstlisting}
\begin{figure}[!h]
\begin{center}
   \includegraphics[width=0.8\linewidth]{Figures/StablishConnectionServer}
	\decoRule
	\caption[Request/Replay conexión Servidor]{Request/Replay conexión al Servidor.}
\label{fig:EjecucionStablishConnection}
\end{center}
\end{figure}
Tras haber enviado el mensaje \textbf{New\_Joined} definimos el evento \textbf{message} para gestionar los mensajes de señalizacion entre navegadores ya que el servidor es transparente a este proceso.
\begin{lstlisting}[
caption=Mensajes de señalizacion.]
 socket.on('message',function(message,room){
  io.sockets.socket(message.id_dest).emit('message', message);
 });
\end{lstlisting}
La figura \ref{fig:AnswerServer} encamina la oferta, la figura \ref{fig:OfferServer} encamina la oferta y la figura \ref{fig:IceCandidateVideos} encamina los IceCandidate.
\begin{figure}[!h]
\begin{center}
   \includegraphics[width=0.8\linewidth]{Figures/AnswerServer}
	\decoRule
	\caption[Request/Replay conexión Servidor]{Mensaje tipo Answer.}
\label{fig:AnswerServer}
\end{center}
\end{figure}
\begin{figure}[!h]
\begin{center}
   \includegraphics[width=0.8\linewidth]{Figures/OfferServer}
	\decoRule
	\caption[Request/Replay conexión Servidor]{Mensaje tipo Offer.}
\label{fig:OfferServer}
\end{center}
\end{figure}
\begin{figure}[!h]
\begin{center}
   \includegraphics[width=0.8\linewidth]{Figures/IceCandidateVideos}
	\decoRule
	\caption[Request/Replay conexión Servidor]{Mensaje tipo Icecandidate.}
\label{fig:IceCandidateVideos}
\end{center}
\end{figure}
El servidor deja de participar en el intercambio de información entre los peers cuando el proceso de señalizacion termina ya que desde ese momento la comunicación es Peer-to-Peer.
%%%%%%%% Cliente Desarrollo %%%%%%%%
\section{Desarrollo Cliente Peer-to-Peer}
Los usuarios se conectan al servidor a través de la url \textbf{http://localhost:8181} en un navegador que recibe el fichero \textbf{index.html}\footnote{Apéndice 4: Index}.
\\Se establece la conexión con el servidor a través de Websockets y se pide al usuario que introduzca el nombre que utilizara en la aplicación.
\begin{lstlisting}[
caption=Instancia WebSockect en el cliente.]
 var socket = io.connect("http://localhost:8181");
\end{lstlisting}
\subsection*{Mensajes de sala}
Cuando el usuario accede a la aplicación envía un mensaje \textbf{infoRoom} a través de WebSockets al servidor para obtener la lista de salas disponibles, figura \ref{fig:SelectItemsClient}.
\begin{lstlisting}[
caption=Petición salas disponibles.]
 socket.emit('infoRoom');
\end{lstlisting}
Se establece el evento \textbf{ReplayInfoRoom} para recibir la lista de salas disponibles. Esta información se añade a la pagina, figura \ref{fig:SelectItemsClient}.
\begin{lstlisting}[
caption=Creación desplegable de salas.]
 socket.on('ReplayInfoRoom',function(listRoom){
  for(var i = 0; i < listRoom.length; i++) {
   var room = listRoom[i];
   $('#listRoom').append('<li><a id='+room.name+'>'+room.name+'</a></li>');
   $('#'+room.name).click(function(){
     nameRoom = $(this).text();
     attachmentElements();
   });
  }
 });
\end{lstlisting}
\subsection*{Establecimiento de conexión}
 El usuario dispone en la barra de navegación de la posibilidad de seleccionar los elementos que desea compartir, figura \ref{fig:SelectItemsClient}.
\begin{figure}[!h]
\centering
\includegraphics[width=0.8\linewidth]{Figures/SelectItemsClient}
\decoRule
\caption[Petición/Recepción salas.]{Petición/Recepción salas.}
\label{fig:SelectItemsClient}
\end{figure}
\\Una vez los elementos multimedia a compartir han sido seleccionados el usuario puede unirse a una de las salas existentes o crear una nueva sala.
\\Con cualquiera de estas opciones se envía un mensaje \textbf{stablish\_conection} con el nombre del usuario y el de la sala a la que desea conectarse, figura \ref{fig:StablishConnectionClient}.
\begin{lstlisting}[
caption=Envió mensaje inicio conexión.]
 socket.emit('stablish_connection',name,nameRoom);
\end{lstlisting}
Para tratar la respuesta definimos el evento \textbf{CreateStream} con el id de la conexión, figura \ref{fi:StablishConnectionClient}.
\begin{lstlisting}[
caption=Recepción respuesta inicio conexión.]
 socket.on('CreateStream',function(id){
  my_id = id;
 });
\end{lstlisting}
\begin{figure}[!h]
\centering
\includegraphics[width=0.8\linewidth]{Figures/StablishConnectionClient}
\decoRule
\caption[Petición/Respuesta establecimiento Conexión.]{Petición/Respuesta establecimiento Conexión.}
\label{fig:StablishConnectionClient}
\end{figure}
Mientras que los demás usuarios recibirán el mensaje 'New\_Joined' con el identificador asociado al usuario que ha solicitado la conexión.
\begin{lstlisting}[
caption=Incluir elementos multimedia remotos l.]
 socket.on('New_Joined',function(id){
  id_newUser = id;
  create_connection(id_newUser);
 });
\end{lstlisting}
A partir de este mensaje empieza el proceso de señalizacion. Este proceso se divide en tres etapas:Offert,Answer y Icecandidate.
%%%%%%%% Cliente offert %%%%%%%%
\subsection*{Offert}
La función \textbf{create\_connection} utiliza el identificador del nuevo cliente e inicia el proceso de oferta.Primero definimos la configuración del  protocolo ICE mediante la variable \textbf{pc\_config} y generamos instancia del objeto \textbf{RTCPeerConnection()} que se almacena en la variable \textbf{pc}.
\begin{lstlisting}[
caption=Instancia RTCPeerConnection.]
 var pc_config = {'iceServers': [{'url': 'stun:stun.l.google.com:19302'}]};
 var pc = new RTCPeerConnection(pc_config,{});
 var num_user = 'user_'+ list_user.length;
 new_remote(num_user);
\end{lstlisting}
Tras esto, es necesario generar una nueva etiqueta de vídeo para tratar el vídeo remoto una vez establecida la conexión.
\begin{lstlisting}[
caption=Creación tag vídeo remoto.]
function new_remote(num_user){
 $('#list_remote').append('<video id='+num_user+'></video');
}
\end{lstlisting}
A continuación, por medio de la variable \textbf{pc} accedemos al método \textbf{addStream()} al que le pasamos el flujo de vídeo local y al evento \textbf{onaddstream} que se encargara de presentar el flujo de vídeo remoto, figura \ref{fig:OfferCliente}.
\begin{lstlisting}[
caption=Vinculamos vídeo local/remoto a RTCPeerConnection.]
 /* video local */
 pc.addStream(streaming);
 /*  video remoto*/
 pc.onaddstream = function(event){
  var video = document.querySelector('#'+num_user);
  video.mozSrcObject = event.stream;
  video.play();
 };
\end{lstlisting}
El siguiente paso es definir el canal de comunicación de datos a través del método \textbf{pc.createDataChannel} al que se le pasa el nombre del canal y definimos los eventos necesarios para manejar los mensajes, figura \ref{fig:DataChannelOffert}.
\begin{lstlisting}[
caption=Instancia de DataChannel.]
 /* canal de datos */
 var sendChannel = pc.createDataChannel("sendDataChannel",{reliable: true})
 /* guardamos el canal */
 list_send.push(sendChannel);	
 /* eventos manejo de datos */
 sendChannel.onopen = ChannelOpen;
 sendChannel.onclose = ChannelClose;
 sendChannel.onmessage = ChannelReceive;
\end{lstlisting}
\begin{figure}[!h]
\centering
\includegraphics[width=0.8\linewidth]{Figures/DataChannelOffert}
\decoRule
\caption[Creación canal de datos cliente.]{Creación canal de datos cliente.}
\label{fig:DataChannelOffert}
\end{figure}
Finalmente, definimos el método \textbf{pc.createOffer()} en el que guardamos la descripción de sesión local con el método \textbf{pc.setLocalDescription(sessionDescription)} y enviamos un mensaje \textbf{message} donde el cuerpo del mensaje es la oferta generada, figura \ref{fig:OfferCliente}.
\begin{lstlisting}[
caption=Creación de la oferta.]
 pc.createOffer(function(sessionDescription){
  //guardamos esto en nuestra session
  pc.setLocalDescription(sessionDescription);
  //enviamos nuestra descripcion al nuevo usuario
  var message = create_msg(my_id,id_newUser,sessionDescription);
  socket.emit('message',message);
 },function(err){console.log(err);},{});
\end{lstlisting}
\begin{figure}[!h]
\centering
\includegraphics[width=0.8\linewidth]{Figures/OfferCliente}
\decoRule
\caption[Creación oferta cliente.]{Creación oferta cliente.}
\label{fig:OfferCliente}
\end{figure}

%%%%%%%% Cliente Answer %%%%%%%%
\subsection*{Answer}
El cliente que origina la oferta recibe un mensaje de tipo \textbf{message} en el que evaluamos el subtipo de mensaje ya que a través de este tipo de mensaje recibimos los distintos tipos de mensaje de señalizacion.
\\Al principio generamos una instancia de \textbf{RTCPeerconection()} de la misma forma que se realizo en la oferta. A continuación, con la información recibida se genera un nuevo objeto \textbf{RTCSessionDescription} que se guarda como descripción de sesión remota a través del método \textbf{setRemoteDescription()}. 
\begin{lstlisting}[
caption=Guardar descripción de sesión remota.]
 pc.setRemoteDescription(new RTCSessionDescription(message.message));
\end{lstlisting}
El siguiente paso es establecer el canal de comunicación que en este caso tenemos que utilizar el evento \textbf{ondatachannel} ya que solo se puede crear un canal de comunicación entre dos nodos. De igual forma que en la oferta creamos las funciones para los eventos del canal, figura \ref{fig:DataChannelAnswer}. 
\begin{lstlisting}[
caption=Creación de canal de recepción de datos.]
 pc.ondatachannel = function(event){
  list_send.push(event.channel);
  var receiveChannel = event.channel;
  /* evento de recepcion */
  receiveChannel.onmessage = ChannelReceive;
  receiveChannel.onopen = ChannelOpen;
  receiveChannel.onclose = ChannelClose;
 }
\end{lstlisting}
\begin{figure}[!h]
\centering
\includegraphics[width=0.8\linewidth]{Figures/DataChannelAnswer}
\decoRule
\caption[Creación canal de recepción datos.]{Creación canal de recepción datos.}
\label{fig:DataChannelAnswer}
\end{figure}
Finalmente, creamos la respuesta a la oferta a través del método \textbf{createAnswer()} dentro del cual guardaremos la información de nuestra propia sesión a través del método \textbf{setLocalDescription} y enviamos un mensaje de tipo \textbf{message} con la información de la sesión local para que el nodo remoto guarde esta información.
\begin{lstlisting}[
caption=Creación de la respuesta.]
 pc.createAnswer(function(sessionDescription){
  pc.setLocalDescription(sessionDescription);
  var msg = create_msg(my_id,message.id_origen,sessionDescription);
  socket.emit('message',msg);
 },function(err){console.log(err);},{});
\end{lstlisting}
\begin{figure}[!h]
\centering
\includegraphics[width=0.8\linewidth]{Figures/AnswerCliente}
\decoRule
\caption[Creación de la respuesta.]{Creación de la respuesta.}
\label{fig:Conexcion_finish}
\end{figure}

%%%%%%%% Cliente IceCandidate %%%%%%%%
\subsection*{Icecandidate}
Tanto el usuario local y como el remoto tras la generación del objeto 'RTCPeerconnectio()' necesitan obtener información de la \textbf{Ip:Puerto} disponibles para la conexión por medio del protocolo ICE.
\\Cuando se encuentra un candidato se ejecuta el evento \textbf{onicecandidate} que compone el mensaje con la información red  y lo envía a través de un mensaje de tipo \textbf{message}.
\begin{lstlisting}[
caption=Envió candidatos.]
 pc.onicecandidate = SendICecandidate;
 ...............
 function SendICecandidate(event){
  if(event.candidate){
   var ice = {type: 'iceCandidate',
    label: event.candidate.sdpMLineIndex,
    id: event.candidate.sdpMid,
    candidate: event.candidate.candidate
   };
   var msg ={id_origen:my_id,id_dest:id_newUser,message:ice}
   socket.emit('message',msg);
  }
}
\end{lstlisting}
Los usuarios que reciben la información anterior generan un objeto \textbf{RTCIceCandidate()} y se llama a la función \textbf{addIcecandidate} que se encarga de buscar dentro de la lista de conexiones la correspondiente y así guardar el objeto creado a través del método \textbf{addIceCandidate}.
\begin{lstlisting}[
caption=Recepción de candidatos.]
 var candidate = new RTCIceCandidate({sdpMLineIndex:message.message.label,
  candidate:message.message.candidate
 });
 addIceCandid(message.id_origen,candidate);
 .............
 function addIceCandid(id,message){
  for(var i=0;i<list_user.length;i++){
   var user = list_user[i];
   if(user.id == id){
    user.peer.addIceCandidate(message);
   }
  }
 }
\end{lstlisting}
Una vez finalizado el proceso de señalizacion la comunicación pasa a ser Peer-to-Peer entre los clientes de una sala, figura \ref{fig:Coneccion_finish}.
\begin{figure}[!h]
\centering
\includegraphics[width=0.5\linewidth]{Figures/Conexcion_finish}
\decoRule
\caption[Conexión Peer-to-Peer final.]{Conexión Peer-to-Peer final.}
\label{fig:Coneccion_finish}
\end{figure}
\subsection*{Envió de cadena de texto}
A través del chat de la sala se puede enviar cadena de caracteres entre los usuarios  mediante la función \textbf{send\_data()}. La función se encarga de obtener los caracteres que el usuario a tecleado y genera un mensaje que contiene el flag \textbf{chat} y el dato obtenido mediante el canal de comunicación, figura \ref{fig:ChatClienteSend}.
\begin{lstlisting}[
caption=Envió datos del chat.]
function send_data(elemento){
  var msg = $(elemento).val();
  var data = JSON.stringify({info:'chat',data:name+':'+msg});
  for(var i=0;i<list_send.length;i++){
   var user = list_send[i];
   user.send(data);
  }
  $(elemento).val('');
}
\end{lstlisting}
\begin{figure}[!h]
\centering
\includegraphics[width=0.8\linewidth]{Figures/ChatClienteSend}
\decoRule
\caption[Envió de caracteres del chat por RTCDataChannel.]{Envió de caracteres del chat por RTCDataChannel.}
\label{fig:ChatClienteSend}
\end{figure}
La recepción por parte de los usuarios se realiza por medio de la función \textbf{WriteChat(\_data)} que se encarga de presentar dentro del chat la información que se ha recibido.
\begin{lstlisting}[
caption=Recepción datos del fichero.]
function WriteChat(_data){
 var line = document.createElement('li');
 var textnode = document.createTextNode(_data.data);
 line.appendChild(textnode);
 $('#texto').append(line);
}
\end{lstlisting}
\subsection*{Envió de ficheros}
Los usuarios disponen de un \textbf{input} de tipo file con el que carga el fichero que desea compartir.Tras seleccionar el fichero se ejecuta la función \textbf{processFiles(file)} para obtener el contenido del fichero a través del objeto \textbf{FileReader()}, figura \ref{fig:fildSendUser}. 
\begin{lstlisting}[
caption=Lectura del fichero.]
 function processFiles(file){
  var files = file[0];
  type = files.type;
  name_fich = files.name;
  var reader = new FileReader();
  reader.onload = function (e) {
   var data_file = reader.result;
   data_encript = arrayBufferToBase64(data_file);
   send_chucky();
  };
  reader.readAsArrayBuffer(files);
 }
\end{lstlisting}
El envió de los datos se realiza por medio de la función \textbf{send\_chucky()} en pequeños fragmentos de longitud fija ya que no sabemos la longitud del archivo y con el fin de no saturar el canal lo enviamos de esta forma.
\\Cada envió por el canal esta formado por el flag \textbf{file} y el fragmento del archivo correspondiente una vez se ha enviado se programa el siguiente envió mediante el evento timer \textbf{setTimeout(sendChuncky,time)}.
\\Al enviar el fragmento final del fichero se añade información adicional del fichero como el nombre y tipo de fichero ya que esta información es necesario para que el usuario receptor pueda reconstruir el fichero, figura \ref{fig:fildSendUser}.
\begin{lstlisting}[
caption=Envió de fragmentos del archivo.]
 function send_chucky(){
  var last = false;
  fin = inicio + size_data;
  if(fin < data_encript.length){
   var data = JSON.stringify({info:'file',data:data_encript.slice(inicio, fin)});
   for(var i=0;i<list_send.length;i++){
    var user = list_send[i];
    user.send(data);
   }
   inicio = fin;
   setTimeout(send_chucky, 100);
  }else{
   last = true;
   var more_info ={type:type,name:name_fich};
   var data = JSON.stringify({info:'file',end:last,data:data_encript.slice(inicio, data_encript.length),more:more_info});
   for(var i=0;i<list_send.length;i++){
    var user = list_send[i];
    user.send(data);
   }
   inicio = 0;
  }
}
\end{lstlisting}
\begin{figure}[!h]
\centering
\includegraphics[width=0.8\linewidth]{Figures/filSendUser}
\decoRule
\caption[Envió información del fichero con  RTCDataChannel.]{Envió información del fichero con  RTCDataChannel.}
\label{fig:fildSendUser}
\end{figure}
El usuario receptor ira acumulando cada uno de los fragmentos que reciba hasta obtener el ultimo fragmento mediante la función \textbf{BuildField()}. Al obtener el ultimo fragmento pasamos a generar el documento a través una etiqueta '<a>' donde el atributo href esta formado por el tipo de archivo concadenado a los fragmentos del archivos, figura \ref{fig:FieldReceiveUser}.
\begin{lstlisting}[
caption=Recepción y reconstrucción del fichero .]
 function BuildFile(_data) {
  blob += _data.data;
  if(_data.end != undefined){
   if(_data.more.type == 'text/plain'){
    var link = document.createElement('a');
    link.href = 'data:'+_data.more.type+';base64'+blob;
    link.target = '_blank';
    link.download = _data.more.name;
    var textnode = document.createTextNode(_data.more.name);
    link.appendChild(textnode);
    $("#listFile").append(link);
   }
   blob =',';
  }
 }  
\end{lstlisting}
\begin{figure}[!h]
\centering
\includegraphics[width=0.8\linewidth]{Figures/filReceiveUser}
\decoRule
\caption[Recepción y reconstrucción del fichero]{Recepción y reconstrucción del fichero.}
\label{fig:FieldReceiveUser}
\end{figure}
\section{Pruebas}
